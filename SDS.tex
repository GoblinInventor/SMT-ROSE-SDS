\documentclass[11pt]{article}
\setlength{\topmargin}{-.5in}
\setlength{\textheight}{9in}
\setlength{\oddsidemargin}{.125in}
\setlength{\textwidth}{6.25in}
\begin{document}
\title{Software Design Document for ROSE SMT Interface}
\author{Michael Hoffman\\
ROSE Compiler Group LLNL\\
\renewcommand{\today}{March 2, 2016}
\maketitle

\section{Introduction}
This is the SDS for the SMTPathFeasibility project using the ROSE Compiler. The program will translate the ROSE AST (Abstract Syntax Tree) and CFG (Control Flow Graph) to a form usable by SMT Solvers under a general solving standard with an abstract base class for different solver architectures.
\section{Goals}
The SMTPathFeasibility project should have the following characteristics:\\
\begin{enumeration}
\item Easily extensible to other solvers
\begin{enumeration}
\item Separation of solver from ROSE analysis
\item Attributes common to solvers in an abstract base class Solver
\item Allow solvers to be used even if they have limited solving capability (or are specially tailored). 
\item Base class specifications of virtual functions should be limited enough to keep a specific implementation as straightforward as possible. Extra capabilities may be very helpful (especially in the case of mu-Z3 where loops are more easily described, see mu-Z3 and Loops for further information 
\end{enumeration}
\end{enumeration}
\section{Coverage Necessities}
\section{Types}
\section{ANSI C Types and Values And ROSE SgNode}
\subsection{ANSI C SgTypes Supported}
\begin{enumeration}
\item char - SgTypeChar
\item signed char - SgTypeSignedChar
\item unsigned char - SgTypeUnsignedChar
\item short - SgTypeShort
\item short int - SgTypeShort
\item signed short - SgTypeSignedShort
\item signed short int - SgTypeSignedShort
\item unsigned short - SgTypeUnsignedShort
\item unsigned short int - SgTypeUnsignedShort
\item int - SgTypeInt
\item signed - SgTypeSignedInt
\item signed int - SgTypeSignedInt
\item unsigned - SgTypeUnsignedInt
\item unsigned int - SgTypeUnsignedInt
\item long - SgTypeLong
\item long int - SgTypeLong
\item signed long - SgTypeSignedLong
\item signed long int - SgTypeSignedLong
\item unsigned long - SgTypeUnsignedLong
\item unsigned long int - SgTypeUnsignedLong
\item long long - SgTypeLongLong
\item long long int - SgTypeLongLong
\item signed long long - SgTypeSignedLongLong
\item signed long long int - SgTypeSignedLongLong
\item unsigned long long - SgTypeUnsignedLongLong
\item unsigned long long int - SgTypeUnsignedLongLong
\item float - SgTypeFloat
\item double - SgTypeDouble
\item long double - SgTypeLongDouble
\item enum - SgEnumType (SgNamedType)
\item union - SgNa
\item arrays (1) - SgArrayType
\item functions (1) - SgFunctionType
\item pointers (2) - SgPointerType
\item struct - SgClassType (SgNamedType)
\item union - SgClassType (SgNamedType)
\item labels - SgTypeLabel
\item void - SgTypeVoid
\end{enumerate}
\subsection{SgTypes Not Supported}
\begin{enumerate}
\item SgDeclType
\item SgJavaUnionType
\item SgModifierType
\item SgQualifiedNameType
\item SgReferenceType
\item SgRvalueReferenceType
\item SgTemplateType
\item SgTypeBool
\item SgTypeCAFTeam
\item SgTypeComplex
\item SgTypeCrayPointer
\item SgTypeDefault
\item SgTypeEllipse
\item SgTypeGlobalVoid
\item SgTypeImaginary
\item SgTypeMatrix
\item SgTypeNullptr
\item SgTypeOfType
\item SgTypeSigned128bitInteger
\item SgTypeString
\item SgTypeTuple
\item SgTypeUnknown
\item SgTypeUnsigned128bitInteger
\item SgTypeWchar
\end{enumerate}
TODO: Check enumeration correctness
\subsection{Value Expression Nodes Supported}
\begin{enumerate}
\item SgCharVal
\item SgDoubleVal
\item SgEnumVal
\item SgFloatVal
\item SgIntVal
\item SgLongDoubleVal
\item SgLongIntVal
\item SgLongLongIntVal
\item SgShortVal  
\item SgUnsignedCharVal
\item SgUnsignedIntVal
\item SgUnsignedLongVal
\item SgUnsignedShortVal
\end{enumerate}
\subsection{Value Nodes Not Supported}
\begin{enumerate}
\item SgBoolValExp
\item SgComplexVal
\item SgNullptrValExp
\item SgStringVal
\item SgTemplateParameterVal
\item SgUpcMythread
\item SgUpcThreads
\item SgWcharVal
\end{enumerate}

\section{Expressions and SMT}
\subsection{Expression Nodes Supported}
\begin{enumerate}
\item SgBinaryOp - see Binary Op Nodes Supported
\item SgClassExp - struct and union
\item SgConditionalExp - ternary "test ? true : false"
\item SgExprListExp 
\item SgFunctionParameterRefExp
\item SgFunctionRefExp
\item SgInitializerExp - see Initializer Exp Nodes Supported
\item SgLabelRefExp
\item SgSizeOfOp
\item SgStatementExpression
\item SgUnaryOp -- see Unary Ops Supported
\item SgValueExp -- see Value Expressions Supported
\item SgVarRefExp 

\end{enumerate}

\subsection{Expression Nodes Not Supported}
\begin{enumerate}
\item SgActualArgumentExpression
\item SgAlignOfOp
\item SgAsmOp
\item SgAsterixShapeExp
\item SgCAFCoExpression
\item SgCallExpression - C++ only SgFunctionCallExp
\item SgClassNameRefExp
\item SgColonShapeExp
\item SgCompoundLiteralExp
\item SgComprehension
\item SgCudaKernelExecConfig
\item SgDeleteExp
\item SgDictionaryComprehension
\item SgDictionaryExp
\item SgHereExp
\item SgImpliedDo
\item SgIOItemExpression
\item SgJavaAnnotation
\item SgJavaInstanceOfOp
\item SgJavaTypeExpression
\item SgKeyDatumPair
\item SgListComprehension
\item SgMagicColonExp
\item SgMemberFunctionRefExp - need to check this
\item SgNaryOp
\item SgNewExp
\item SgNoexceptOp
\item SgNullExpression - need to check this
\item SgPseudoDestructorRefExp
\item SgRangeExp
\item SgSetComprehension
\item SgSubscriptExpression
\item SgSuperExp
\item SgTemplateFunctionRefExp
\item SgTemplateFunctionMemberRefExp
\item SgThisExp
\item SgTypeExpression - need to check this
\item SgTypeIdOp
\item SgTypeTraitBuiltinOperator
\item SgUnknownArrayOrFunctionReference
\item SgUpcBlocksizeofExpression
\item SgUpcLocalsizeExpression
\item SgVarArgCopyOp
\item SgVarArgEndOp
\item SgVarArgOp
\item SgVariantExpression - need to check this
\item SgYieldExpression - need to check this
\end{enumerate}


\subsection{Binary Op Nodes Supported}
\begin{enumerate}
\item SgAddOp
\item SgAndOp
\item SgArrowExp
\item SgAssignOp
\item SgBitAndOp
\item SgBitOrOp
\item SgBitXorOp
\item SgDivideOp
\item SgEqualityOp
\item SgGreaterOrEqualOp
\item SgGreaterThanOp
\item SgIntegerDivideOp
\item SgLessOrEqualOp
\item SgLessThanOp
\item SgLshiftOp
\item SgModOp
\item SgMultiplyOp
\item SgNotEqualOp
\item SgOrOp
\item SgPntrArrRefExp
\item SgRshiftOp
\item SgSubtractOp

\end{enumerate}

\subsection{Binary Op Nodes Not Supported}
\begin{enumerate}
\item SgCommaOpExp
\item SgConcatenationOp
\item SgDotExp
\item SgDotStarOp
\item SgElementwiseOp
\item SgExponentiationOp
\item SgIsNotOp
\item SgIsOp
\item SgJavaUnsignedRshiftOp
\item SgLeftDivideOp
\item SgMembershipOp
\item SgNonMembershipOp
\item SgPointerAssignOp
\item SgPowerOp
\item SgScopeOp - Deprecated
\item SgUserDefinedBinaryOp
\end{enumerate}

\subsection{Compound Assign Op Nodes Supported}
\begin{enumerate}
\item SgAndAssignOp
\item SgDivAssignOp
\item SgLshiftAssignOp
\item SgMinusAssignOp
\item SgModAssignOp
\item SgPlusAssignOp
\item SgRshiftAssignOp
\item SgXorAssignOp
\end{enumerate}

\subsetion{Compound Assign Op Nodes Not Supported}
\begin{enumerate}
\item SgExponentiationAssignOp
\item SglorAssignOp
\item SgJavaUnsignedRshiftAssignOp
\end{enumerate}

\subsection{Initializer Nodes Supported}
\begin{enumerate}
\item SgAggregateInitializer - "int x[2] = {1,2}"
\item SgAssignInitializer - "int x = 1", 1 wrapped in SgAssignInializer
\end{enumerate}

\subsection{Initializer Nodes Not Supported}
\begin{enumerate}
\item SgCompoundAssignInitializer - need to check this
\item SgConstructorInitializer - never used
\item SgDesignatedInitializer - need to check this
\end{enumerate}

\subsection{Unary Ops Supported}
\begin{enumerate}
\item SgAddressOfOp
\item SgBitComplementOp
\item SgCastExp
\item SgMinusMinusOp
\item SgMinusOp
\item SgNotOp
\item SgPlusPlusOp
\item SgPointerDerefExp
\end{enumerate}

\subsection{Unary Ops Not Supported}
\begin{enumerate}
\item SgConjugateOp
\item SgExpressionRoot - need to check this
\item SgImagPartOf
\item SgMatrixTransposeOp
\item SgRealPartOp
\item SgThrowOp
\item SgUnaryAddOp -- need to check this
\item SgUserDefinedUnaryOp
\end{enumerate}

\subsection{Statements Supported}
\begin{enumerate}
\item SgBreakStmt
\item SgCaseOptionStmt
\item SgContinueStmt
\item SgDeallocationStatment
\item SgDeclarationStatment -- see Supported Declaration Statements
\item SgDefaultOptionStmt
\item SgForInitStatement
\item SgGotoStatement
\item SgLabelStatement
\item SgReturnStmt

\end{enumerate}

\subsection{Statements Not Supported}
\begin{enumerate}
\item SgAllocateStatement
\item SgArithmeticIf
\item SgAssertStatement
\item SgAsyncStmt
\item SgAtStmt
\item SgCatchStatementSeq
\item SgComputedGotoStatement
\item SgElseWhereStatement
\item SgExecStatement
\item SgFinishStmt
\item SgFunctionTableType
\item SgIOStatement
\item SgJavaSynchronizedStatement
\item SgJavaThrowStatement
\item SgNullifyStatement
\item SgNullStatement
\item SgOmpBarrierStatement
\item SgOmpBodyStatement
\item SgOmpFlushStatement
\item SgPassStatement
\item SgPythonGlobalStmt
\item SgPythonPrintStmt
\item SgSequenceStatement
\item SgSpawnStmt
\item SgStopOrPauseStatement
\item SgTryStmt
\item SgUpcBarrierStatement
\item SgUpcFenceStatement
\item SgUpcNotifyStatement
\item SgUpcWaitStatement
\item SgVariantStatement
\item SgWhereStatement
\item SgWithStatement
\end{enumerate}

\subsection{Scope Statements Supported}
\begin{enumerate}
\item SgBasicBlock
\item SgCaseOptionStmt
\item SgClassDefinition -- see classes in C
\item SgDoWhileStmt
\item SgForStatement
\item SgFunctionDefinition
\item SgIfStmt
\item SgSwitchStatement
\item SgWhileStmt
\end{enumerate}

\subsection{Scope Statements Not Supported}
\begin{enumerate}
\item SgAssociateStatement
\item SgBlockDataStatement
\item SgCAFWithTeamStatement
\item SgForAllStatement
\item SgFortranDo
\item SgFunctionPaameterScope
\item SgGlobal
\item SgJavaForEachStatement
\item SgJavaLabelStatement
\item SgMatlabForStatement
\item SgNamespaceDefinitionStatement
\item SgUpcForAllStatement
\end{enumerate}


\subsection{Declarations Supported}
\begin{enumerate}
\item SgClassDeclaration -- see Class Declaration Node in C
\item SgEnumDeclaration
\item SgFunctionDeclaration -- see Function Declarations Supported
\item SgFunctionParameterList
\item SgParameterStatement
\item SgPragmaDeclaration
\item SgTypedefDeclaration
\item SgVariableDeclaration
\item SgVariableDefinition
\end{enumerate}


\subsection{Declarations Not Supported}
\begin{enumerate}
\item SgAsmStmt
\item SgAttributeSpecificationStatement
\item SgC\_PreprocessorDirectiveStatement
\item SgClinkageDeclarationStatement
\item SgCommonBlock
\item SgContainsStatement
\item SgCtorInitializerList
\item SgEquivalenceStatement
\item SgFormatStatement
\item SgFortranIncludeLine
\item SgImplicitStatement
\item SgImportStatement
\item SgInterfaceStatement
\item SgJavaImportStatement
\item SgJavaPackageStatement
\item SgMicrosoftAttributeDeclaration
\item SgNamelistStatement
\item SgNamespaceAliasDeclarationStatement
\item SgNamespaceDeclarationStatement
\item SgOmpThreadprivateStatement
\item SgStatementFunctionStatement
\item SgStaticAssertionDeclaration
\item SgStmtDeclarationStatement
\item SgTemplateDeclaration
\item SgTemplateInstantiationDirectiveStatement 
\item SgUseStatement
\item SgUsingDeclarationStatement
\item SgUsingDirectiveStatement
\end{enumerate}

\section{Following Yacc Grammar For Translation to SMT Forms}
SOURCE: https://www.lysator.liu.se/c/ANSI-C-grammar-y.html#assignment-expression
Starting at statement:
\section{Statements}
\subsection{Labeled Statement}
labeled\_statement:
Concept is not represented in ROSE as per\\
http://rosecompiler.org/ROSE\_HTML\_Reference/classSgLabelStatement.html\\
\subsection{Compound Statement}
Just a statement list or a declaration list, this should not be a problem for the solver because the translator preparing info for the solver does not recognize a compound\_statement as a specific unit, instead it is deconstructed into individual statements and given to the solver in that form.\\
\subsection{Expression Statement}
expression\_statement:
We can access the interior expression via
get\_expression
Expressions are handled later in this document
\subsection{Selection Statement}
selection\_statement:
We're dealing with SgIfStmt or an SgSwitchStatement
\subsubsection{If Statement}
In the if statement we're dealing with 
IF '(' expression ')' statement
or
IF '(' expression ')' statement ELSE statement
Given SgIfStmt, we can retrieve information as follows as per ROSE documentation\\
conditional expression: use
get\_conditional()
which returns SgStatement* which should be a SgExprStatement which will be handled as an expression by SgExprStatement with
get\_expression()
which returns an SgExpression* that is potentially empty. We need to check that it is not the case, and for now that will cause a failure
We can access the body of the if and else via
get\_true\_body()
and
get\_false\_body()
each of which return an SgStatement, if both bodies exist, so we must check for the existence of the false\_body. According to the ROSE documentation these should specifically return an SgBasicBlock, so this assumption should be made assertion, potentially only in a debug mode, and any abnormalities can be dealt with. The association of false and true body needs to be explicitly stated so that the solver can easily form the conditional value for a specific path correctly.
\subsubsection{Switch Statement}
In the case of a switch statement we can get the item selector via
get\_item\_selector
which returns an SgExpression, which we can evaluate and the conditionals can be properly expressed. For example, if we have the relatively common cases without breaks we need to concatenate them with an or. This may cause a problem if short circuit evaluation is used so we will need to account for that. The body of the switch is accessible by
get\_body()  
In a normative case switch construct we will find
SgCaseOptionStmt
we can get the key for the statement via
get\_key()
which returns an expression statement and we can get the body via
get\_body()
which returns an SgBasicBlock
\subsection{Iteration Statement: For, Do/While, While}
The method for analyzing loops is noted in "Fixed Point Methods For Analysis" and "Alternatives to Fixed Point Methods"
For extraction of necessary information
\subsubsection{For Loops}
As per the yacc grammar the For Loop is of the for
FOR '(' expression\_statement expression\_statement ')' statement
or
FOR '(' expression\_statement expression\_statement expression ')' statement
From ROSE we get the first expression statement via
get\_for\_init\_stmt()
We then use
get\_init\_stmt()
on the SgForInitStmt to get a 
SgStatementPtrList\&
which is a typedef of STL list, so we can manipulate that via
empty
to check if the for init is empty
begin
to get the iterator (call it ITR) to the beginning and iterate through via ++ITR
We then transform each statement into its SMT compatible class
\subsection{Jump Statements}
Jump statements (break, continue, goto) need to be considered within their own scope. This is handled in Fixed Point Methods For Analysis.
\section{Expressions}


\section{Type Promotion}
NOT YET DONE



\section{mu-Z3 and Loops}
NOT YET DONE
\section{Fixed Point Methods for Analysis}
NOT YET DONE
\section{mu-Z3 and the typical solver necessities}
NOT YET DONE
\section{Generalizability of Fixed Point Methods}
NOT YET DONE
\setion{Alternatives to Fixed Point Methods}
\end{document}


