\documentclass[11pt]{article}
\setlength{\topmargin}{-.5in}
\setlength{\textheight}{9in}
\setlength{\oddsidemargin}{.125in}
\setlength{\textwidth}{6.25in}
\begin{document}
\title{Software Design Document for ROSE SMT Interface}
\author{Michael Hoffman\\
ROSE Compiler Group LLNL\\
\renewcommand{\today}{March 2, 2016}
\maketitle

\section{Introduction}
This is the SDS for the SMTPathFeasibility project using the ROSE Compiler. The program will translate the ROSE AST (Abstract Syntax Tree) and CFG (Control Flow Graph) to a form usable by SMT Solvers under a general solving standard with an abstract base class for different solver architectures.
\section{Goals}
The SMTPathFeasibility project should have the following characteristics:\\
\begin{enumeration}
\item Easily extensible to other solvers
\begin{enumeration}
\item Separation of solver from ROSE analysis
\item Attributes common to solvers in an abstract base class Solver
\item Allow solvers to be used even if they have limited solving capability (or are specially tailored). 
\item Base class specifications of virtual functions should be limited enough to keep a specific implementation as straightforward as possible. Extra capabilities may be very helpful (especially in the case of mu-Z3 where loops are more easily described, see mu-Z3 and Loops for further information 
\end{enumeration}
\end{enumeration}
\section{Coverage Necessities}
\section{Types}
\section{ANSI C Types and Values And ROSE SgNode}
\subsection{ANSI C SgTypes Supported}
\begin{enumeration}
\item char - SgTypeChar
\item signed char - SgTypeSignedChar
\item unsigned char - SgTypeUnsignedChar
\item short - SgTypeShort
\item short int - SgTypeShort
\item signed short - SgTypeSignedShort
\item signed short int - SgTypeSignedShort
\item unsigned short - SgTypeUnsignedShort
\item unsigned short int - SgTypeUnsignedShort
\item int - SgTypeInt
\item signed - SgTypeSignedInt
\item signed int - SgTypeSignedInt
\item unsigned - SgTypeUnsignedInt
\item unsigned int - SgTypeUnsignedInt
\item long - SgTypeLong
\item long int - SgTypeLong
\item signed long - SgTypeSignedLong
\item signed long int - SgTypeSignedLong
\item unsigned long - SgTypeUnsignedLong
\item unsigned long int - SgTypeUnsignedLong
\item long long - SgTypeLongLong
\item long long int - SgTypeLongLong
\item signed long long - SgTypeSignedLongLong
\item signed long long int - SgTypeSignedLongLong
\item unsigned long long - SgTypeUnsignedLongLong
\item unsigned long long int - SgTypeUnsignedLongLong
\item float - SgTypeFloat
\item double - SgTypeDouble
\item long double - SgTypeLongDouble
\item enum - SgEnumType (SgNamedType)
\item union - SgNa
\item arrays (1) - SgArrayType
\item functions (1) - SgFunctionType
\item pointers (2) - SgPointerType
\item struct - SgClassType (SgNamedType)
\item union - SgClassType (SgNamedType)
\item labels - SgTypeLabel
\item void - SgTypeVoid
\end{enumerate}
\subsection{SgTypes Not Supported}
\begin{enumerate}
\item SgDeclType
\item SgJavaUnionType
\item SgModifierType
\item SgQualifiedNameType
\item SgReferenceType
\item SgRvalueReferenceType
\item SgTemplateType
\item SgTypeBool
\item SgTypeCAFTeam
\item SgTypeComplex
\item SgTypeCrayPointer
\item SgTypeDefault
\item SgTypeEllipse
\item SgTypeGlobalVoid
\item SgTypeImaginary
\item SgTypeMatrix
\item SgTypeNullptr
\item SgTypeOfType
\item SgTypeSigned128bitInteger
\item SgTypeString
\item SgTypeTuple
\item SgTypeUnknown
\item SgTypeUnsigned128bitInteger
\item SgTypeWchar
\end{enumerate}
TODO: Check enumeration correctness
\subsection{Value Nodes Supported}
\begin{enumerate}
\item SgCharVal
\item SgDoubleVal
\item SgEnumVal
\item SgFloatVal
\item SgIntVal
\item SgLongDoubleVal
\item SgLongIntVal
\item SgLongLongIntVal
\item SgShortVal  
\item SgUnsignedCharVal
\item SgUnsignedIntVal
\item SgUnsignedLongVal
\item SgUnsignedShortVal
\end{enumerate}
\subsection{Value Nodes Not Supported}
\begin{enumerate}
\item SgBoolValExp
\item SgComplexVal
\item SgNullptrValExp
\item SgStringVal
\item SgTemplateParameterVal
\item SgUpcMythread
\item SgUpcThreads
\item SgWcharVal
\end{enumerate}
\section{Expressions and SMT}
\subsection{Expression Nodes Supported}
\begin{enumerate}
\item 

\subsection{Expression Nodes Not Supported}
\begin{enumerate}
\item SgActualArgumentExpression
\item SgAlignOfOp


\section{Following Yacc Grammar For Translation to SMT Forms}
SOURCE: https://www.lysator.liu.se/c/ANSI-C-grammar-y.html#assignment-expression
Starting at statement:
labeled\_statement:
Concept is not represented in ROSE as per\\
http://rosecompiler.org/ROSE\_HTML\_Reference/classSgLabelStatement.html\\
compound\_statement:
Just a statement list or a declaration list, this should not be a problem for the solver because the translator preparing info for the solver does not recognize a compound\_statement as a specific unit, instead it is deconstructed into individual statements and given to the solver in that form.\\
expression\_statement:
Expressions are handled later in this document
selection\_statement:
We're dealing with SgIfStmt or an SgSwitchStatement
In the if statement we're dealing with 
IF '(' expression ')' statement
or
IF '(' expression ')' statement ELSE statement
Given SgIfStmt, we can retrieve information as follows as per ROSE documentation\\
conditional expression: use
get\_conditional()
which returns SgStatement* which should be a SgExprStatement which will be handled as an expression by SgExprStatement with
get\_expression()
which returns an SgExpression* that is potentially empty. We need to check that it is not the case, and for now that will cause a failure



\section{Type Promotion}
NOT YET DONE



\section{mu-Z3 and Loops}
NOT YET DONE
\section{Fixed Point Methods for Analysis}
NOT YET DONE
\section{mu-Z3 and the typical solver necessities}
NOT YET DONE
\section{Generalizability of Fixed Point Methods}
NOT YET DONE
\end{document}


